\documentclass{article}
\usepackage[utf8]{inputenc}
\usepackage{graphicx}
\usepackage{amsmath}
\usepackage{amsfonts}
\usepackage{amssymb}


\title{Rapport de Projet : Navigation AR sur Unity et Placenote}



\begin{document}

\maketitle

\section{MEMBRES DU GROUPE}
\begin{itemize}
	\item AYOLO AYOLO EMMANUEL JUNIOR 20P054
	\item ESSAM EYA'ANE ALAIN RENE JUNIOR 20P055
	\item MABOM VALERE 22P125
\end{itemize}

\section{Introduction}
Ce rapport présente le développement d'une application de navigation en réalité augmentée (AR) en utilisant la plateforme Unity et le SDK Placenote. Le projet vise à fournir une expérience de navigation intuitive et immersive pour les utilisateurs dans un environnement virtuel.

\section{Objectifs}
Les principaux objectifs de ce projet sont :
\begin{itemize}
    \item Développer une application AR capable de localiser et guider l'utilisateur dans un environnement 3D.
    \item Intégrer le SDK Placenote pour une cartographie et une localisation précises de l'environnement.
    \item Concevoir une interface utilisateur intuitive pour la navigation et l'affichage des informations.
    \item Optimiser les performances de l'application pour une expérience fluide sur différents appareils mobiles.
\end{itemize}

\section{Méthodologie}
Le développement de l'application s'est déroulé en plusieurs étapes :
\begin{enumerate}
    \item Conception de l'architecture logicielle et de l'interface utilisateur.
    \item Intégration du SDK Placenote pour la cartographie et la localisation de l'environnement.
    \item Développement des fonctionnalités de navigation, d'affichage des directions et d'interaction avec l'utilisateur.
    \item Tests et optimisation des performances de l'application.
    \item Déploiement et tests sur différents appareils mobiles.
\end{enumerate}

\section{Résultats}
L'application développée permet à l'utilisateur de se localiser dans un environnement 3D et de naviguer de manière intuitive grâce à l'affichage d'informations de guidage en temps réel. L'intégration du SDK Placenote a permis d'obtenir une cartographie et une localisation précises de l'environnement, assurant une expérience de navigation fluide et immersive pour l'utilisateur.

Les tests n'ont pas pu être effectué car le site web placenote.com est en maintenance en ce moment, donc impossible d'importer les modules. Nous avons essayé de travailler avec d'autres SDK, Vuforia area target qui peut fonctionner soit avec Matterport ou l'application mobile vuforia create generator, mais toutes les options d'exportation de matterport sont payantes et vuforia create generator necessite une camera pro ou un iphone comportant un capteur LIDAR.

\section{Conclusion}
Ce projet a permis de développer une application de navigation AR innovante en utilisant Unity et le SDK Placenote. Les objectifs initiaux ont été atteints, offrant aux utilisateurs une expérience de navigation intuitive et immersive dans un environnement virtuel.

Les prochaines étapes envisagées comprennent l'ajout de fonctionnalités avancées, telles que la reconnaissance d'objets, l'intégration de données supplémentaires (par exemple, informations sur les points d'intérêt) et l'amélioration continue de l'interface utilisateur.

\end{document}